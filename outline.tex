\documentclass[12pt]{article}
\usepackage[breaklinks=true]{hyperref}
\usepackage{color}
\usepackage{amsmath,amssymb,amsthm}
\usepackage{natbib}
\usepackage[margin=0.75in]{geometry}
\usepackage[singlespacing]{setspace}
\usepackage[bottom]{footmisc}
\usepackage{floatrow}
\usepackage{float,graphicx}
\usepackage{enumerate}


\newtheorem{theorem}{Theorem}[section]
\newtheorem{lemma}[theorem]{Lemma}
\newtheorem{assumption}{Assumption}

\newcommand{\beq}{\begin{equation}}
\newcommand{\eeq}{\end{equation}}


\newcommand{\todo}[1]{{\color{red}{TO DO: \sc #1}}}

\newcommand{\reals}{\mathbb{R}}
\newcommand{\integers}{\mathbb{Z}}
\newcommand{\naturals}{\mathbb{N}}
\newcommand{\rationals}{\mathbb{Q}}

\newcommand{\ind}{\mathbb{I}} % Indicator function
\newcommand{\pr}{\mathbb{P}} % Generic probability
\newcommand{\ex}{\mathbb{E}} % Generic expectation
\newcommand{\var}{\textrm{Var}}
\newcommand{\cov}{\textrm{Cov}}

\newcommand{\normal}{N} % for normal distribution 
\newcommand{\eps}{\varepsilon}
\newcommand\independent{\protect\mathpalette{\protect\independenT}{\perp}}
\def\independenT#1#2{\mathrel{\rlap{$#1#2$}\mkern2mu{#1#2}}}
\newcommand{\argmax}{\textrm{argmax}}
\newcommand{\argmin}{\textrm{argmin}}

\title{Notes: Multivariate Paired Permutation Tests}
\author{Kellie Ottoboni}
\date{\today}
\begin{document}
\maketitle

%\newpage

%\begin{abstract}


%\end{abstract}

%\newpage


\section{Ordered variables}
\begin{itemize}
\item Suppose that we have two paired samples and binary outcomes, and we wish to test whether the frequency of outcomes is the same in the two samples.
For example, we may be interested in whether the incidence of heart disease decreased after a cohort took some treatment.
The McNemar test accomplishes this.
We observe $N$ pairs.
The data may be written in a $2 \times 2$ contingency table, of the form

\begin{table}[h]
\begin{tabular}{|l|c|c|}
\hline
 & Sample 1 + & Sample 1 - \\
 \hline
 Sample 2 + & a & b \\
 \hline 
 Sample 2 - & c & d \\
 \hline
\end{tabular}
\end{table}
where ``+'' and ``-'' denote the binary outcomes in each sample,
$a$ of the observations have ``+'' outcome in both samples,
$b$ of the observations have ``+'' outcome in sample 1 and ``-'' in sample 2, and so forth,
with $a + b + c + d = N$. \\

Let $(X_{i1}, X_{i2})$ denote the $i$th pair of outcomes coming from sample 1 and sample 2, respectively.
McNemar tests

\begin{align*}
H_0&: \pr(X_{\cdot 1} = x) =  \pr(X_{\cdot 2} = x), x = 1, 2 \\
H_1&: \pr(X_{\cdot 1}  = x) \neq  \pr(X_{\cdot 2}  = x), \text{for some } x = 1, 2 
\end{align*}

The null further implies that $\pr(X_{\cdot 1} = x \mid X_{\cdot 2} = y) = \pr(X_{\cdot 2} = x \mid X_{\cdot 1} = y)$
Therefore, $\pr((X_{i1}, X_{i2}) = (x, y)) = \pr((X_{i1}, X_{i2}) = (y, x))$ for all $(x, y) \in \{ +, - \}^2$.
This means that within pairs, observations are exchangeable.


The test statistic is

$$\frac{(b-c)^2}{b+c}$$
To obtain the permutation distribution of the test statistic, we randomly exchange the order of observations within pairs, with probability $1/2$, then count the numbers $b^*$ and $c^*$ in the permuted data.
\item 
\end{itemize}
\section{Missing data}
\begin{itemize}
\item informative censoring
\item non-informative censoring
\item exact?
\end{itemize}

%\bibliographystyle{plainnat}
%\bibliography{refs}


\end{document}